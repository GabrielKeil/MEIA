% Options for packages loaded elsewhere
% Options for packages loaded elsewhere
\PassOptionsToPackage{unicode}{hyperref}
\PassOptionsToPackage{hyphens}{url}
\PassOptionsToPackage{dvipsnames,svgnames,x11names}{xcolor}
%
\documentclass[
  letterpaper,
  DIV=11,
  numbers=noendperiod]{scrartcl}
\usepackage{xcolor}
\usepackage{amsmath,amssymb}
\setcounter{secnumdepth}{5}
\usepackage{iftex}
\ifPDFTeX
  \usepackage[T1]{fontenc}
  \usepackage[utf8]{inputenc}
  \usepackage{textcomp} % provide euro and other symbols
\else % if luatex or xetex
  \usepackage{unicode-math} % this also loads fontspec
  \defaultfontfeatures{Scale=MatchLowercase}
  \defaultfontfeatures[\rmfamily]{Ligatures=TeX,Scale=1}
\fi
\usepackage{lmodern}
\ifPDFTeX\else
  % xetex/luatex font selection
\fi
% Use upquote if available, for straight quotes in verbatim environments
\IfFileExists{upquote.sty}{\usepackage{upquote}}{}
\IfFileExists{microtype.sty}{% use microtype if available
  \usepackage[]{microtype}
  \UseMicrotypeSet[protrusion]{basicmath} % disable protrusion for tt fonts
}{}
\makeatletter
\@ifundefined{KOMAClassName}{% if non-KOMA class
  \IfFileExists{parskip.sty}{%
    \usepackage{parskip}
  }{% else
    \setlength{\parindent}{0pt}
    \setlength{\parskip}{6pt plus 2pt minus 1pt}}
}{% if KOMA class
  \KOMAoptions{parskip=half}}
\makeatother
% Make \paragraph and \subparagraph free-standing
\makeatletter
\ifx\paragraph\undefined\else
  \let\oldparagraph\paragraph
  \renewcommand{\paragraph}{
    \@ifstar
      \xxxParagraphStar
      \xxxParagraphNoStar
  }
  \newcommand{\xxxParagraphStar}[1]{\oldparagraph*{#1}\mbox{}}
  \newcommand{\xxxParagraphNoStar}[1]{\oldparagraph{#1}\mbox{}}
\fi
\ifx\subparagraph\undefined\else
  \let\oldsubparagraph\subparagraph
  \renewcommand{\subparagraph}{
    \@ifstar
      \xxxSubParagraphStar
      \xxxSubParagraphNoStar
  }
  \newcommand{\xxxSubParagraphStar}[1]{\oldsubparagraph*{#1}\mbox{}}
  \newcommand{\xxxSubParagraphNoStar}[1]{\oldsubparagraph{#1}\mbox{}}
\fi
\makeatother

\usepackage{color}
\usepackage{fancyvrb}
\newcommand{\VerbBar}{|}
\newcommand{\VERB}{\Verb[commandchars=\\\{\}]}
\DefineVerbatimEnvironment{Highlighting}{Verbatim}{commandchars=\\\{\}}
% Add ',fontsize=\small' for more characters per line
\usepackage{framed}
\definecolor{shadecolor}{RGB}{241,243,245}
\newenvironment{Shaded}{\begin{snugshade}}{\end{snugshade}}
\newcommand{\AlertTok}[1]{\textcolor[rgb]{0.68,0.00,0.00}{#1}}
\newcommand{\AnnotationTok}[1]{\textcolor[rgb]{0.37,0.37,0.37}{#1}}
\newcommand{\AttributeTok}[1]{\textcolor[rgb]{0.40,0.45,0.13}{#1}}
\newcommand{\BaseNTok}[1]{\textcolor[rgb]{0.68,0.00,0.00}{#1}}
\newcommand{\BuiltInTok}[1]{\textcolor[rgb]{0.00,0.23,0.31}{#1}}
\newcommand{\CharTok}[1]{\textcolor[rgb]{0.13,0.47,0.30}{#1}}
\newcommand{\CommentTok}[1]{\textcolor[rgb]{0.37,0.37,0.37}{#1}}
\newcommand{\CommentVarTok}[1]{\textcolor[rgb]{0.37,0.37,0.37}{\textit{#1}}}
\newcommand{\ConstantTok}[1]{\textcolor[rgb]{0.56,0.35,0.01}{#1}}
\newcommand{\ControlFlowTok}[1]{\textcolor[rgb]{0.00,0.23,0.31}{\textbf{#1}}}
\newcommand{\DataTypeTok}[1]{\textcolor[rgb]{0.68,0.00,0.00}{#1}}
\newcommand{\DecValTok}[1]{\textcolor[rgb]{0.68,0.00,0.00}{#1}}
\newcommand{\DocumentationTok}[1]{\textcolor[rgb]{0.37,0.37,0.37}{\textit{#1}}}
\newcommand{\ErrorTok}[1]{\textcolor[rgb]{0.68,0.00,0.00}{#1}}
\newcommand{\ExtensionTok}[1]{\textcolor[rgb]{0.00,0.23,0.31}{#1}}
\newcommand{\FloatTok}[1]{\textcolor[rgb]{0.68,0.00,0.00}{#1}}
\newcommand{\FunctionTok}[1]{\textcolor[rgb]{0.28,0.35,0.67}{#1}}
\newcommand{\ImportTok}[1]{\textcolor[rgb]{0.00,0.46,0.62}{#1}}
\newcommand{\InformationTok}[1]{\textcolor[rgb]{0.37,0.37,0.37}{#1}}
\newcommand{\KeywordTok}[1]{\textcolor[rgb]{0.00,0.23,0.31}{\textbf{#1}}}
\newcommand{\NormalTok}[1]{\textcolor[rgb]{0.00,0.23,0.31}{#1}}
\newcommand{\OperatorTok}[1]{\textcolor[rgb]{0.37,0.37,0.37}{#1}}
\newcommand{\OtherTok}[1]{\textcolor[rgb]{0.00,0.23,0.31}{#1}}
\newcommand{\PreprocessorTok}[1]{\textcolor[rgb]{0.68,0.00,0.00}{#1}}
\newcommand{\RegionMarkerTok}[1]{\textcolor[rgb]{0.00,0.23,0.31}{#1}}
\newcommand{\SpecialCharTok}[1]{\textcolor[rgb]{0.37,0.37,0.37}{#1}}
\newcommand{\SpecialStringTok}[1]{\textcolor[rgb]{0.13,0.47,0.30}{#1}}
\newcommand{\StringTok}[1]{\textcolor[rgb]{0.13,0.47,0.30}{#1}}
\newcommand{\VariableTok}[1]{\textcolor[rgb]{0.07,0.07,0.07}{#1}}
\newcommand{\VerbatimStringTok}[1]{\textcolor[rgb]{0.13,0.47,0.30}{#1}}
\newcommand{\WarningTok}[1]{\textcolor[rgb]{0.37,0.37,0.37}{\textit{#1}}}

\usepackage{longtable,booktabs,array}
\usepackage{calc} % for calculating minipage widths
% Correct order of tables after \paragraph or \subparagraph
\usepackage{etoolbox}
\makeatletter
\patchcmd\longtable{\par}{\if@noskipsec\mbox{}\fi\par}{}{}
\makeatother
% Allow footnotes in longtable head/foot
\IfFileExists{footnotehyper.sty}{\usepackage{footnotehyper}}{\usepackage{footnote}}
\makesavenoteenv{longtable}
\usepackage{graphicx}
\makeatletter
\newsavebox\pandoc@box
\newcommand*\pandocbounded[1]{% scales image to fit in text height/width
  \sbox\pandoc@box{#1}%
  \Gscale@div\@tempa{\textheight}{\dimexpr\ht\pandoc@box+\dp\pandoc@box\relax}%
  \Gscale@div\@tempb{\linewidth}{\wd\pandoc@box}%
  \ifdim\@tempb\p@<\@tempa\p@\let\@tempa\@tempb\fi% select the smaller of both
  \ifdim\@tempa\p@<\p@\scalebox{\@tempa}{\usebox\pandoc@box}%
  \else\usebox{\pandoc@box}%
  \fi%
}
% Set default figure placement to htbp
\def\fps@figure{htbp}
\makeatother





\setlength{\emergencystretch}{3em} % prevent overfull lines

\providecommand{\tightlist}{%
  \setlength{\itemsep}{0pt}\setlength{\parskip}{0pt}}



 


\KOMAoption{captions}{tableheading}
\makeatletter
\@ifpackageloaded{caption}{}{\usepackage{caption}}
\AtBeginDocument{%
\ifdefined\contentsname
  \renewcommand*\contentsname{Table of contents}
\else
  \newcommand\contentsname{Table of contents}
\fi
\ifdefined\listfigurename
  \renewcommand*\listfigurename{List of Figures}
\else
  \newcommand\listfigurename{List of Figures}
\fi
\ifdefined\listtablename
  \renewcommand*\listtablename{List of Tables}
\else
  \newcommand\listtablename{List of Tables}
\fi
\ifdefined\figurename
  \renewcommand*\figurename{Figure}
\else
  \newcommand\figurename{Figure}
\fi
\ifdefined\tablename
  \renewcommand*\tablename{Table}
\else
  \newcommand\tablename{Table}
\fi
}
\@ifpackageloaded{float}{}{\usepackage{float}}
\floatstyle{ruled}
\@ifundefined{c@chapter}{\newfloat{codelisting}{h}{lop}}{\newfloat{codelisting}{h}{lop}[chapter]}
\floatname{codelisting}{Listing}
\newcommand*\listoflistings{\listof{codelisting}{List of Listings}}
\makeatother
\makeatletter
\makeatother
\makeatletter
\@ifpackageloaded{caption}{}{\usepackage{caption}}
\@ifpackageloaded{subcaption}{}{\usepackage{subcaption}}
\makeatother
\usepackage{bookmark}
\IfFileExists{xurl.sty}{\usepackage{xurl}}{} % add URL line breaks if available
\urlstyle{same}
\hypersetup{
  pdftitle={PCA and Outlier Analysis on Machines Data},
  pdfauthor={Grupo},
  colorlinks=true,
  linkcolor={blue},
  filecolor={Maroon},
  citecolor={Blue},
  urlcolor={Blue},
  pdfcreator={LaTeX via pandoc}}


\title{PCA and Outlier Analysis on Machines Data}
\author{Grupo}
\date{2025-12-09}
\begin{document}
\maketitle

\renewcommand*\contentsname{Table of contents}
{
\hypersetup{linkcolor=}
\setcounter{tocdepth}{3}
\tableofcontents
}

\section{Introduction}\label{introduction}

This report walks through a pedagogical analysis of a subset of the
\texttt{machines} data from the \texttt{rrcov} package (rows
\textbf{hp-3000/64} to \textbf{ibm-4331-2}). Goals: - Describe the
variables with both classical and robust summaries. - Run PCA on the raw
scale and on standardized variables; compare explained variance and
choose components that keep ≥95\% of total variance. - Study the impact
of a deliberately injected outlier using classical PCA and a robust PCA
based on the Minimum Covariance Determinant (MCD).

\section{Setup and data}\label{setup-and-data}

\begin{Shaded}
\begin{Highlighting}[]
\CommentTok{\# install.packages(c("rrcov", "robustbase"), repos = "https://cloud.r{-}project.org")}
\FunctionTok{library}\NormalTok{(rrcov)}
\FunctionTok{library}\NormalTok{(robustbase)}

\FunctionTok{data}\NormalTok{(machines)}
\NormalTok{machines\_sub }\OtherTok{\textless{}{-}}\NormalTok{ machines[}\DecValTok{71}\SpecialCharTok{:}\DecValTok{111}\NormalTok{, ]                  }\CommentTok{\# hp{-}3000/64 ... ibm{-}4331{-}2}
\NormalTok{machines\_sub}\SpecialCharTok{$}\NormalTok{machine }\OtherTok{\textless{}{-}} \FunctionTok{rownames}\NormalTok{(machines\_sub)      }\CommentTok{\# keep names for plotting}
\FunctionTok{rownames}\NormalTok{(machines\_sub) }\OtherTok{\textless{}{-}} \ConstantTok{NULL}

\FunctionTok{head}\NormalTok{(machines\_sub, }\DecValTok{3}\NormalTok{)}
\end{Highlighting}
\end{Shaded}

\begin{verbatim}
  MYCT MMIN MMAX CACH CHMIN CHMAX PRP ERP     machine
1   75 2000 8000    8     3    24  62  47  hp-3000/64
2   75 3000 8000    8     3    48  64  54  hp-3000/88
3  175  256 2000    0     3    24  22  20 hp-3000/iii
\end{verbatim}

\begin{Shaded}
\begin{Highlighting}[]
\FunctionTok{summary}\NormalTok{(machines\_sub)}
\end{Highlighting}
\end{Shaded}

\begin{verbatim}
      MYCT             MMIN            MMAX            CACH      
 Min.   :  26.0   Min.   :   96   Min.   :  512   Min.   : 0.00  
 1st Qu.: 140.0   1st Qu.:  768   1st Qu.: 3000   1st Qu.: 0.00  
 Median : 300.0   Median : 1000   Median : 4000   Median : 4.00  
 Mean   : 322.7   Mean   : 2435   Mean   : 9812   Mean   :10.41  
 3rd Qu.: 330.0   3rd Qu.: 2000   3rd Qu.:12000   3rd Qu.: 8.00  
 Max.   :1100.0   Max.   :16000   Max.   :32000   Max.   :64.00  
     CHMIN            CHMAX             PRP           ERP        
 Min.   : 1.000   Min.   :  1.00   Min.   :  6   Min.   : 15.00  
 1st Qu.: 1.000   1st Qu.:  2.00   1st Qu.: 22   1st Qu.: 23.00  
 Median : 1.000   Median : 20.00   Median : 38   Median : 30.00  
 Mean   : 3.171   Mean   : 20.39   Mean   : 80   Mean   : 69.34  
 3rd Qu.: 4.000   3rd Qu.: 24.00   3rd Qu.: 66   3rd Qu.: 57.00  
 Max.   :16.000   Max.   :112.00   Max.   :465   Max.   :361.00  
   machine         
 Length:41         
 Class :character  
 Mode  :character  
                   
                   
                   
\end{verbatim}

Helper functions for trimmed/winsorized means and total/generalized
variance:

\begin{Shaded}
\begin{Highlighting}[]
\NormalTok{winsor\_mean }\OtherTok{\textless{}{-}} \ControlFlowTok{function}\NormalTok{(x, }\AttributeTok{probs =} \FunctionTok{c}\NormalTok{(}\FloatTok{0.05}\NormalTok{, }\FloatTok{0.95}\NormalTok{)) \{}
\NormalTok{  qs }\OtherTok{\textless{}{-}} \FunctionTok{quantile}\NormalTok{(x, probs, }\AttributeTok{names =} \ConstantTok{FALSE}\NormalTok{)}
  \FunctionTok{mean}\NormalTok{(}\FunctionTok{pmin}\NormalTok{(}\FunctionTok{pmax}\NormalTok{(x, qs[}\DecValTok{1}\NormalTok{]), qs[}\DecValTok{2}\NormalTok{]))}
\NormalTok{\}}

\NormalTok{total\_variance }\OtherTok{\textless{}{-}} \ControlFlowTok{function}\NormalTok{(S) }\FunctionTok{sum}\NormalTok{(}\FunctionTok{diag}\NormalTok{(S))}
\NormalTok{generalized\_variance }\OtherTok{\textless{}{-}} \ControlFlowTok{function}\NormalTok{(S) }\FunctionTok{determinant}\NormalTok{(S, }\AttributeTok{logarithm =} \ConstantTok{TRUE}\NormalTok{)}\SpecialCharTok{$}\NormalTok{modulus}
\end{Highlighting}
\end{Shaded}

\section{Exploratory summaries}\label{exploratory-summaries}

\subsection{Classical vs robust
location/scale}\label{classical-vs-robust-locationscale}

\begin{Shaded}
\begin{Highlighting}[]
\NormalTok{num\_vars }\OtherTok{\textless{}{-}}\NormalTok{ machines\_sub[ , }\FunctionTok{setdiff}\NormalTok{(}\FunctionTok{names}\NormalTok{(machines\_sub), }\StringTok{"machine"}\NormalTok{)]}

\NormalTok{stat\_table }\OtherTok{\textless{}{-}} \FunctionTok{data.frame}\NormalTok{(}
  \AttributeTok{variable =} \FunctionTok{names}\NormalTok{(num\_vars),}
  \AttributeTok{mean =} \FunctionTok{sapply}\NormalTok{(num\_vars, mean),}
  \AttributeTok{median =} \FunctionTok{sapply}\NormalTok{(num\_vars, median),}
  \AttributeTok{trimmed\_mean =} \FunctionTok{sapply}\NormalTok{(num\_vars, mean, }\AttributeTok{trim =} \FloatTok{0.1}\NormalTok{),}
  \AttributeTok{winsor\_mean =} \FunctionTok{sapply}\NormalTok{(num\_vars, winsor\_mean),}
  \AttributeTok{sd =} \FunctionTok{sapply}\NormalTok{(num\_vars, sd),}
  \AttributeTok{var =} \FunctionTok{sapply}\NormalTok{(num\_vars, var),}
  \AttributeTok{mad =} \FunctionTok{sapply}\NormalTok{(num\_vars, mad)}
\NormalTok{)}

\NormalTok{stat\_table}
\end{Highlighting}
\end{Shaded}

\begin{verbatim}
      variable        mean median trimmed_mean winsor_mean           sd
MYCT      MYCT  322.707317    300   276.575758  312.951220   301.747845
MMIN      MMIN 2435.121951   1000  1542.787879 2050.341463  3563.887261
MMAX      MMAX 9811.707317   4000  8212.121212 9829.268293 10584.538663
CACH      CACH   10.414634      4     6.151515   10.414634    17.915043
CHMIN    CHMIN    3.170732      1     2.484848    2.878049     3.390446
CHMAX    CHMAX   20.390244     20    16.606061   18.731707    22.381776
PRP        PRP   80.000000     38    54.575758   71.048780   108.868728
ERP        ERP   69.341463     30    50.393939   62.829268    84.751286
               var       mad
MYCT  9.105176e+04  237.2160
MMIN  1.270129e+07 1103.0544
MMAX  1.120325e+08 4447.8000
CACH  3.209488e+02    5.9304
CHMIN 1.149512e+01    0.0000
CHMAX 5.009439e+02   20.7564
PRP   1.185240e+04   32.6172
ERP   7.182780e+03   17.7912
\end{verbatim}

\subsection{Covariance, total and generalized
variance}\label{covariance-total-and-generalized-variance}

\begin{Shaded}
\begin{Highlighting}[]
\NormalTok{S\_classic }\OtherTok{\textless{}{-}} \FunctionTok{cov}\NormalTok{(num\_vars)}
\NormalTok{total\_var }\OtherTok{\textless{}{-}} \FunctionTok{total\_variance}\NormalTok{(S\_classic)}
\NormalTok{gen\_var\_log }\OtherTok{\textless{}{-}} \FunctionTok{generalized\_variance}\NormalTok{(S\_classic)  }\CommentTok{\# log{-}determinant for stability}

\FunctionTok{list}\NormalTok{(}
  \AttributeTok{covariance\_matrix =}\NormalTok{ S\_classic,}
  \AttributeTok{total\_variance =}\NormalTok{ total\_var,}
  \AttributeTok{generalized\_variance\_log =}\NormalTok{ gen\_var\_log}
\NormalTok{)}
\end{Highlighting}
\end{Shaded}

\begin{verbatim}
$covariance_matrix
               MYCT         MMIN         MMAX         CACH        CHMIN
MYCT     91051.7622  -432377.463  -1609363.49  -2092.07561  -440.823780
MMIN   -432377.4634 12701292.410  24690893.74  41241.87317  8756.653659
MMAX  -1609363.4878 24690893.737 112032458.71 138025.22439 17389.951220
CACH     -2092.0756    41241.873    138025.22    320.94878    33.402439
CHMIN     -440.8238     8756.654     17389.95     33.40244    11.495122
CHMAX    -2618.3579     3924.551     92296.42    110.28415     4.831707
PRP     -15881.2000   359417.800    968283.00   1556.55000   277.225000
ERP     -11618.7226   263724.107    820483.00   1253.72988   187.415244
             CHMAX        PRP         ERP
MYCT  -2618.357927 -15881.200 -11618.7226
MMIN   3924.551220 359417.800 263724.1073
MMAX  92296.417073 968283.000 820483.0024
CACH    110.284146   1556.550   1253.7299
CHMIN     4.831707    277.225    187.4152
CHMAX   500.943902    442.150    481.2134
PRP     442.150000  11852.400   8981.1000
ERP     481.213415   8981.100   7182.7805

$total_variance
[1] 124844671

$generalized_variance_log
[1] 67.78829
attr(,"logarithm")
[1] TRUE
\end{verbatim}

\subsection{Mahalanobis distances
(classical)}\label{mahalanobis-distances-classical}

\begin{Shaded}
\begin{Highlighting}[]
\NormalTok{md }\OtherTok{\textless{}{-}} \FunctionTok{mahalanobis}\NormalTok{(num\_vars, }\AttributeTok{center =} \FunctionTok{colMeans}\NormalTok{(num\_vars), }\AttributeTok{cov =}\NormalTok{ S\_classic)}
\NormalTok{cutoff }\OtherTok{\textless{}{-}} \FunctionTok{qchisq}\NormalTok{(}\FloatTok{0.975}\NormalTok{, }\AttributeTok{df =} \FunctionTok{ncol}\NormalTok{(num\_vars))}

\FunctionTok{plot}\NormalTok{(md, }\AttributeTok{pch =} \DecValTok{19}\NormalTok{, }\AttributeTok{main =} \StringTok{"Mahalanobis distances"}\NormalTok{, }\AttributeTok{ylab =} \StringTok{"Distance"}\NormalTok{)}
\FunctionTok{abline}\NormalTok{(}\AttributeTok{h =}\NormalTok{ cutoff, }\AttributeTok{col =} \StringTok{"red"}\NormalTok{, }\AttributeTok{lty =} \DecValTok{2}\NormalTok{)}
\end{Highlighting}
\end{Shaded}

\pandocbounded{\includegraphics[keepaspectratio]{report_files/figure-pdf/unnamed-chunk-6-1.pdf}}

\section{Principal Component Analysis (original vs
standardized)}\label{principal-component-analysis-original-vs-standardized}

\begin{Shaded}
\begin{Highlighting}[]
\NormalTok{pca\_raw }\OtherTok{\textless{}{-}} \FunctionTok{prcomp}\NormalTok{(num\_vars, }\AttributeTok{center =} \ConstantTok{TRUE}\NormalTok{, }\AttributeTok{scale. =} \ConstantTok{FALSE}\NormalTok{)}
\NormalTok{pca\_std }\OtherTok{\textless{}{-}} \FunctionTok{prcomp}\NormalTok{(num\_vars, }\AttributeTok{center =} \ConstantTok{TRUE}\NormalTok{, }\AttributeTok{scale. =} \ConstantTok{TRUE}\NormalTok{)}

\NormalTok{pve\_raw }\OtherTok{\textless{}{-}}\NormalTok{ pca\_raw}\SpecialCharTok{$}\NormalTok{sdev}\SpecialCharTok{\^{}}\DecValTok{2} \SpecialCharTok{/} \FunctionTok{sum}\NormalTok{(pca\_raw}\SpecialCharTok{$}\NormalTok{sdev}\SpecialCharTok{\^{}}\DecValTok{2}\NormalTok{)}
\NormalTok{pve\_std }\OtherTok{\textless{}{-}}\NormalTok{ pca\_std}\SpecialCharTok{$}\NormalTok{sdev}\SpecialCharTok{\^{}}\DecValTok{2} \SpecialCharTok{/} \FunctionTok{sum}\NormalTok{(pca\_std}\SpecialCharTok{$}\NormalTok{sdev}\SpecialCharTok{\^{}}\DecValTok{2}\NormalTok{)}
\end{Highlighting}
\end{Shaded}

\subsection{Scree plots and variance
explained}\label{scree-plots-and-variance-explained}

\begin{Shaded}
\begin{Highlighting}[]
\FunctionTok{par}\NormalTok{(}\AttributeTok{mfrow =} \FunctionTok{c}\NormalTok{(}\DecValTok{1}\NormalTok{, }\DecValTok{2}\NormalTok{))}
\FunctionTok{plot}\NormalTok{(pve\_raw }\SpecialCharTok{*} \DecValTok{100}\NormalTok{, }\AttributeTok{type =} \StringTok{"b"}\NormalTok{, }\AttributeTok{pch =} \DecValTok{19}\NormalTok{, }\AttributeTok{xlab =} \StringTok{"PC"}\NormalTok{, }\AttributeTok{ylab =} \StringTok{"\% variance"}\NormalTok{,}
     \AttributeTok{main =} \StringTok{"Raw scale"}\NormalTok{)}
\FunctionTok{lines}\NormalTok{(}\FunctionTok{cumsum}\NormalTok{(pve\_raw) }\SpecialCharTok{*} \DecValTok{100}\NormalTok{, }\AttributeTok{type =} \StringTok{"b"}\NormalTok{, }\AttributeTok{col =} \StringTok{"blue"}\NormalTok{)}
\FunctionTok{abline}\NormalTok{(}\AttributeTok{h =} \DecValTok{95}\NormalTok{, }\AttributeTok{col =} \StringTok{"red"}\NormalTok{, }\AttributeTok{lty =} \DecValTok{2}\NormalTok{)}

\FunctionTok{plot}\NormalTok{(pve\_std }\SpecialCharTok{*} \DecValTok{100}\NormalTok{, }\AttributeTok{type =} \StringTok{"b"}\NormalTok{, }\AttributeTok{pch =} \DecValTok{19}\NormalTok{, }\AttributeTok{xlab =} \StringTok{"PC"}\NormalTok{, }\AttributeTok{ylab =} \StringTok{"\% variance"}\NormalTok{,}
     \AttributeTok{main =} \StringTok{"Standardized"}\NormalTok{)}
\FunctionTok{lines}\NormalTok{(}\FunctionTok{cumsum}\NormalTok{(pve\_std) }\SpecialCharTok{*} \DecValTok{100}\NormalTok{, }\AttributeTok{type =} \StringTok{"b"}\NormalTok{, }\AttributeTok{col =} \StringTok{"blue"}\NormalTok{)}
\FunctionTok{abline}\NormalTok{(}\AttributeTok{h =} \DecValTok{95}\NormalTok{, }\AttributeTok{col =} \StringTok{"red"}\NormalTok{, }\AttributeTok{lty =} \DecValTok{2}\NormalTok{)}
\end{Highlighting}
\end{Shaded}

\pandocbounded{\includegraphics[keepaspectratio]{report_files/figure-pdf/unnamed-chunk-8-1.pdf}}

\begin{Shaded}
\begin{Highlighting}[]
\FunctionTok{par}\NormalTok{(}\AttributeTok{mfrow =} \FunctionTok{c}\NormalTok{(}\DecValTok{1}\NormalTok{, }\DecValTok{1}\NormalTok{))}
\end{Highlighting}
\end{Shaded}

\begin{Shaded}
\begin{Highlighting}[]
\NormalTok{k\_raw }\OtherTok{\textless{}{-}} \FunctionTok{which}\NormalTok{(}\FunctionTok{cumsum}\NormalTok{(pve\_raw) }\SpecialCharTok{\textgreater{}=} \FloatTok{0.95}\NormalTok{)[}\DecValTok{1}\NormalTok{]}
\NormalTok{k\_std }\OtherTok{\textless{}{-}} \FunctionTok{which}\NormalTok{(}\FunctionTok{cumsum}\NormalTok{(pve\_std) }\SpecialCharTok{\textgreater{}=} \FloatTok{0.95}\NormalTok{)[}\DecValTok{1}\NormalTok{]}

\FunctionTok{data.frame}\NormalTok{(}
  \AttributeTok{scale =} \FunctionTok{c}\NormalTok{(}\StringTok{"raw"}\NormalTok{, }\StringTok{"standardized"}\NormalTok{),}
  \AttributeTok{pcs\_needed\_for\_95pct =} \FunctionTok{c}\NormalTok{(k\_raw, k\_std),}
  \AttributeTok{cumulative\_variance =} \FunctionTok{c}\NormalTok{(}\FunctionTok{cumsum}\NormalTok{(pve\_raw)[k\_raw], }\FunctionTok{cumsum}\NormalTok{(pve\_std)[k\_std])}
\NormalTok{)}
\end{Highlighting}
\end{Shaded}

\begin{verbatim}
         scale pcs_needed_for_95pct cumulative_variance
1          raw                    2           0.9994522
2 standardized                    5           0.9724443
\end{verbatim}

\subsection{Loadings and interpretation
aids}\label{loadings-and-interpretation-aids}

\begin{Shaded}
\begin{Highlighting}[]
\FunctionTok{head}\NormalTok{(pca\_raw}\SpecialCharTok{$}\NormalTok{rotation[, }\DecValTok{1}\SpecialCharTok{:}\FunctionTok{min}\NormalTok{(}\DecValTok{3}\NormalTok{, }\FunctionTok{ncol}\NormalTok{(num\_vars))])}
\end{Highlighting}
\end{Shaded}

\begin{verbatim}
                PC1           PC2           PC3
MYCT   0.0141412358 -0.0077399303  0.9994839149
MMIN  -0.2286221980  0.9731769922  0.0106566935
MMAX  -0.9733452865 -0.2289360796  0.0119681990
CACH  -0.0012202199  0.0012445172  0.0001322595
CHMIN -0.0001606719  0.0006590092 -0.0020541284
CHMAX -0.0007701548 -0.0025019794 -0.0219924018
\end{verbatim}

\begin{Shaded}
\begin{Highlighting}[]
\FunctionTok{head}\NormalTok{(pca\_std}\SpecialCharTok{$}\NormalTok{rotation[, }\DecValTok{1}\SpecialCharTok{:}\FunctionTok{min}\NormalTok{(}\DecValTok{3}\NormalTok{, }\FunctionTok{ncol}\NormalTok{(num\_vars))])}
\end{Highlighting}
\end{Shaded}

\begin{verbatim}
             PC1         PC2         PC3
MYCT   0.2609775 -0.39938737 -0.72811374
MMIN  -0.3860768 -0.28892323  0.06954451
MMAX  -0.3871015  0.16725096 -0.28349782
CACH  -0.3699990  0.01382335 -0.32360056
CHMIN -0.3328552 -0.25326137  0.45179763
CHMAX -0.1386368  0.79963560 -0.15685676
\end{verbatim}

\subsection{Scores plot (first two
PCs)}\label{scores-plot-first-two-pcs}

\begin{Shaded}
\begin{Highlighting}[]
\NormalTok{scores\_std }\OtherTok{\textless{}{-}} \FunctionTok{as.data.frame}\NormalTok{(pca\_std}\SpecialCharTok{$}\NormalTok{x)}
\NormalTok{scores\_std}\SpecialCharTok{$}\NormalTok{machine }\OtherTok{\textless{}{-}}\NormalTok{ machines\_sub}\SpecialCharTok{$}\NormalTok{machine}

\FunctionTok{plot}\NormalTok{(scores\_std}\SpecialCharTok{$}\NormalTok{PC1, scores\_std}\SpecialCharTok{$}\NormalTok{PC2, }\AttributeTok{pch =} \DecValTok{19}\NormalTok{,}
     \AttributeTok{xlab =} \StringTok{"PC1 (std)"}\NormalTok{, }\AttributeTok{ylab =} \StringTok{"PC2 (std)"}\NormalTok{,}
     \AttributeTok{main =} \StringTok{"Scores on standardized data"}\NormalTok{)}
\FunctionTok{text}\NormalTok{(scores\_std}\SpecialCharTok{$}\NormalTok{PC1, scores\_std}\SpecialCharTok{$}\NormalTok{PC2, }\AttributeTok{labels =}\NormalTok{ scores\_std}\SpecialCharTok{$}\NormalTok{machine,}
     \AttributeTok{pos =} \DecValTok{3}\NormalTok{, }\AttributeTok{cex =} \FloatTok{0.6}\NormalTok{)}
\end{Highlighting}
\end{Shaded}

\pandocbounded{\includegraphics[keepaspectratio]{report_files/figure-pdf/unnamed-chunk-11-1.pdf}}

\section{Outlier experiment}\label{outlier-experiment}

Introduce the atypical point at the former \texttt{hp-3000/64} row (no
standardization).

\begin{Shaded}
\begin{Highlighting}[]
\NormalTok{xnew }\OtherTok{\textless{}{-}} \FunctionTok{c}\NormalTok{(}\DecValTok{75}\NormalTok{, }\DecValTok{2000}\NormalTok{, }\FloatTok{0.8}\NormalTok{, }\DecValTok{80000}\NormalTok{, }\DecValTok{300}\NormalTok{, }\DecValTok{24}\NormalTok{, }\DecValTok{62}\NormalTok{, }\DecValTok{47}\NormalTok{)}
\NormalTok{machines\_out }\OtherTok{\textless{}{-}}\NormalTok{ machines\_sub}
\NormalTok{machines\_out[machines\_out}\SpecialCharTok{$}\NormalTok{machine }\SpecialCharTok{==} \StringTok{"hp{-}3000/64"}\NormalTok{, }\FunctionTok{names}\NormalTok{(num\_vars)] }\OtherTok{\textless{}{-}}\NormalTok{ xnew}

\NormalTok{pca\_out\_classic }\OtherTok{\textless{}{-}} \FunctionTok{prcomp}\NormalTok{(machines\_out[}\FunctionTok{names}\NormalTok{(num\_vars)], }\AttributeTok{center =} \ConstantTok{TRUE}\NormalTok{, }\AttributeTok{scale. =} \ConstantTok{FALSE}\NormalTok{)}
\NormalTok{pca\_out\_robust }\OtherTok{\textless{}{-}} \FunctionTok{PcaCov}\NormalTok{(machines\_out[}\FunctionTok{names}\NormalTok{(num\_vars)], }\AttributeTok{cov.control =} \FunctionTok{CovControlMcd}\NormalTok{(), }\AttributeTok{scale =} \ConstantTok{FALSE}\NormalTok{)}

\NormalTok{pve\_out\_classic }\OtherTok{\textless{}{-}}\NormalTok{ pca\_out\_classic}\SpecialCharTok{$}\NormalTok{sdev}\SpecialCharTok{\^{}}\DecValTok{2} \SpecialCharTok{/} \FunctionTok{sum}\NormalTok{(pca\_out\_classic}\SpecialCharTok{$}\NormalTok{sdev}\SpecialCharTok{\^{}}\DecValTok{2}\NormalTok{)}
\NormalTok{pve\_out\_robust }\OtherTok{\textless{}{-}}\NormalTok{ pca\_out\_robust}\SpecialCharTok{@}\NormalTok{eigenvalues }\SpecialCharTok{/} \FunctionTok{sum}\NormalTok{(pca\_out\_robust}\SpecialCharTok{@}\NormalTok{eigenvalues)}

\FunctionTok{data.frame}\NormalTok{(}
  \AttributeTok{component =} \FunctionTok{seq\_along}\NormalTok{(pve\_out\_classic),}
  \AttributeTok{classic\_pct =} \FunctionTok{round}\NormalTok{(pve\_out\_classic }\SpecialCharTok{*} \DecValTok{100}\NormalTok{, }\DecValTok{2}\NormalTok{),}
  \AttributeTok{robust\_pct =} \FunctionTok{round}\NormalTok{(pve\_out\_robust }\SpecialCharTok{*} \DecValTok{100}\NormalTok{, }\DecValTok{2}\NormalTok{)}
\NormalTok{)}
\end{Highlighting}
\end{Shaded}

\begin{verbatim}
  component classic_pct robust_pct
1         1       57.96      89.76
2         2       39.58       7.87
3         3        2.44       2.36
4         4        0.02       0.00
5         5        0.00       0.00
6         6        0.00       0.00
7         7        0.00       0.00
8         8        0.00       0.00
\end{verbatim}

\subsection{Visual comparison: scores}\label{visual-comparison-scores}

\begin{Shaded}
\begin{Highlighting}[]
\FunctionTok{par}\NormalTok{(}\AttributeTok{mfrow =} \FunctionTok{c}\NormalTok{(}\DecValTok{1}\NormalTok{, }\DecValTok{2}\NormalTok{))}
\FunctionTok{plot}\NormalTok{(pca\_out\_classic}\SpecialCharTok{$}\NormalTok{x[,}\DecValTok{1}\NormalTok{], pca\_out\_classic}\SpecialCharTok{$}\NormalTok{x[,}\DecValTok{2}\NormalTok{], }\AttributeTok{pch =} \DecValTok{19}\NormalTok{,}
     \AttributeTok{main =} \StringTok{"Classic PCA with outlier"}\NormalTok{, }\AttributeTok{xlab =} \StringTok{"PC1"}\NormalTok{, }\AttributeTok{ylab =} \StringTok{"PC2"}\NormalTok{)}
\FunctionTok{text}\NormalTok{(pca\_out\_classic}\SpecialCharTok{$}\NormalTok{x[,}\DecValTok{1}\NormalTok{], pca\_out\_classic}\SpecialCharTok{$}\NormalTok{x[,}\DecValTok{2}\NormalTok{],}
     \AttributeTok{labels =}\NormalTok{ machines\_out}\SpecialCharTok{$}\NormalTok{machine, }\AttributeTok{pos =} \DecValTok{3}\NormalTok{, }\AttributeTok{cex =} \FloatTok{0.5}\NormalTok{)}

\FunctionTok{plot}\NormalTok{(pca\_out\_robust}\SpecialCharTok{@}\NormalTok{scores[,}\DecValTok{1}\NormalTok{], pca\_out\_robust}\SpecialCharTok{@}\NormalTok{scores[,}\DecValTok{2}\NormalTok{], }\AttributeTok{pch =} \DecValTok{19}\NormalTok{,}
     \AttributeTok{main =} \StringTok{"Robust PCA (MCD)"}\NormalTok{, }\AttributeTok{xlab =} \StringTok{"PC1"}\NormalTok{, }\AttributeTok{ylab =} \StringTok{"PC2"}\NormalTok{)}
\FunctionTok{text}\NormalTok{(pca\_out\_robust}\SpecialCharTok{@}\NormalTok{scores[,}\DecValTok{1}\NormalTok{], pca\_out\_robust}\SpecialCharTok{@}\NormalTok{scores[,}\DecValTok{2}\NormalTok{],}
     \AttributeTok{labels =}\NormalTok{ machines\_out}\SpecialCharTok{$}\NormalTok{machine, }\AttributeTok{pos =} \DecValTok{3}\NormalTok{, }\AttributeTok{cex =} \FloatTok{0.5}\NormalTok{)}
\end{Highlighting}
\end{Shaded}

\pandocbounded{\includegraphics[keepaspectratio]{report_files/figure-pdf/unnamed-chunk-13-1.pdf}}

\begin{Shaded}
\begin{Highlighting}[]
\FunctionTok{par}\NormalTok{(}\AttributeTok{mfrow =} \FunctionTok{c}\NormalTok{(}\DecValTok{1}\NormalTok{, }\DecValTok{1}\NormalTok{))}
\end{Highlighting}
\end{Shaded}

\subsubsection{Distance diagnostics (robust
PCA)}\label{distance-diagnostics-robust-pca}

\begin{Shaded}
\begin{Highlighting}[]
\FunctionTok{plot}\NormalTok{(pca\_out\_robust)  }\CommentTok{\# outlier map: orthogonal vs score distances}
\end{Highlighting}
\end{Shaded}

\pandocbounded{\includegraphics[keepaspectratio]{report_files/figure-pdf/unnamed-chunk-14-1.pdf}}

\section{Decisions and conclusions}\label{decisions-and-conclusions}

\begin{itemize}
\tightlist
\item
  Classical summaries provide a baseline; compare trimmed/winsorized
  means and MAD to flag skew/heavy tails.
\item
  PCA on raw scale vs standardized: choose the variant that reaches
  ≥95\% variance with fewer components and clearer loadings; document
  which variables dominate each retained PC.
\item
  With the injected outlier, classical PCA reorients strongly toward the
  extreme point (variance inflation, first PC dominated by the outlier).
  The MCD-based PCA keeps stable directions and variance shares.
\item
  For reporting: state which PCA you recommend (often standardized if
  scales differ), how many components you keep (based on the 95\% rule),
  and how the outlier analysis supports the robustness choice.
\end{itemize}

\section{Bibliography}\label{bibliography}

\begin{itemize}
\tightlist
\item
  Jolliffe, I. T. (2002). \emph{Principal Component Analysis}. Springer.
\item
  Todorov, V., \& Filzmoser, P. (2009). An object oriented framework for
  robust multivariate analysis. \emph{Journal of Statistical Software},
  32(3), 1--47. ***
\end{itemize}




\end{document}
